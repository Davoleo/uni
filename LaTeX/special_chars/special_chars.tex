\documentclass[a4paper]{article}
\usepackage[utf8]{inputenc}
\usepackage{xcolor}
\usepackage{soul}
\usepackage{hyperref}

\author{Leonardo Davoli}
\title{Special Characters in \LaTeX}
\date{\today}

%% BEGIN: Custom Commands %%

\newcommand{\mystyle}[1]{\hl{\textit{\textbf{\texttt{#1}}}}}
\newcommand{\mystylecolor}[2]{\sethlcolor{#1} \hl{\textit{\textbf{\texttt{#2}}}}}

%% END: Custom Commands %%

\begin{document}

	\maketitle
	\tableofcontents

	\section{Caratteri speciali e i loro usi in \LaTeX}

	\subsection{Space [\ ]}
	This      sentence      contains various   spaces,
	although every group of consecutive spaces is collapsed into a single one.
	Inserting a new line character is also considered like a space in the compiled documents,
	allowing to break lines in the source independently from the compiled document structure.

	If you insert 2 new line characters it's considered as if you're beginning 
	to write a new paragraph which actually begins a new line and also indents it as per standard conventions.

	\subsection{Double Quote ["]}
	``Example quoted part of text", can also be used as a command in that case it has the
	function of adding a special double pointed accent to the character \"{T}

	\subsection{Hashtag [\#]}
	Has the function of aligning things horizontally, can be escaped by writing: \textbackslash\#

	\subsection{Dollar [\${}]}
	Used to open small in-line math environments to write formulas and calculations,
	for example: $ e^{i\pi} = -1 $

	Can be escaped by writing: \textbackslash\$

	\subsection{Percentage [\%]}
	Used to write comments in \LaTeX .tex source files,
	text after \% is ignored and stripped by the compiler.

	Can be escaped this way: \textbackslash\%

	\subsection{Ampersand [\&]}
	Reserved character to align things vertically, can be escaped: \textbackslash\&

	\subsection{Apostrophe (Single Quote) [']}
	Can be used normally in text to single-quote text: 'Example single-quoted text'

	Can be used as a command to put acute accent to a character: \'{H}

	\subsection{Backslash [\textbackslash]}
	Special Character used to begin any macro command in \LaTeX.

	Can be escaped by writing: \textbackslash textbackslash

	\subsection{Caret [\^{}]}
	\begin{itemize}
		\item Can be used as a normal character in text: \^{}
		\item Can be used to put circumflex accent on a certain character: \^{J}
		\item Can be used to write something in superscript in math environments [e.g. exponents]: $ 2^4 $
	\end{itemize}

	\subsection{Underscore [\_]}
	Used in math environments to write something as subscript of something else,
	for instance: $F_{in} = 2N$ or $F_{out} = 1N$

	Can be escaped by writing: \textbackslash\_

	\subsection{Backtick [`]}
	The inverse symbol of the single quote,
	can be used to add grave accents to characters in this way: \`{F}

	\subsection{Open and Closed Brace [\{ \}]}
	Reserved \LaTeX\ character to pass parameters to commands and environment names.

	Can be escaped: \textbackslash\{, \textbackslash\}

	\subsection{Tilde [\~{}]}
	\begin{itemize}
		\item Used~to~insert~a~space~that~can't~be~broken~when~\LaTeX~wants~to~put~the~words~on~2~different~lines
		\item Can be used to add a tilde accent on a character: \~{Y}
		\item Can be escaped: \~{}
	\end{itemize}

	\section{Commands \& Declarations tests}
	\begin{itemize}
		\item First command that highlights: \hl{Highlit Text}
		\item First command that highlights in green: {\sethlcolor{green} \hl{Highlit Text}}
		\item Back to the original color...? \hl{Highlit Text}
		\item Change style and highlight: \hl{\textit{\textbf{Highlit Text}}}
		\item Change style and highlight: \hl{\textit{\textbf{\texttt{Highlit Text}}}}
		\item Change style and highlight: \hl{\textsc{Highlit Text}}
		\item \href{https://open.spotify.com/track/4AfmZe8i6uNAR5xTSZxqNn?si=aa3887b5fc244fe9}{Questa è una freccia: $\rightarrow$ [cit. Rancore]}
		\item Questa è una frazione: $\frac{num}{den}$
		\item \mystyle{This is my style using custom command}
		\item \mystylecolor{pink}{This is my style but colored :moyai:}
		
	\end{itemize}

	


\end{document}