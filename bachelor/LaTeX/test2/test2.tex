\documentclass[11pt,b5paper,openany,titlepage,twoside]{book}
\usepackage[utf8]{inputenc}

% My packages
\usepackage{geometry}
\usepackage[italian]{babel}
\usepackage{blindtext}
\usepackage{mathtools,amsfonts, amssymb}
\usepackage[hidelinks]{hyperref}
\usepackage[italian,tight]{minitoc}

\hypersetup{
	colorlinks=true,
	linkcolor=blue,
	filecolor=magenta,
	urlcolor=cyan,
	pdftitle=My \LaTeX Document,
	pdfpagemode=FullScreen
}

\title{My \LaTeX\ Document}
\author{Leonardo Davoli}
\date{\today}

\setcounter{secnumdepth}{2}
\setcounter{tocdepth}{2}
\setcounter{minitocdepth}{3}
\dominitoc

\begin{document}
	
	\frontmatter
	\maketitle
	\tableofcontents

	\mainmatter
	\chapter{Background}
	\minitoc
	\mtcskip
	In questo capitolo si presenteranno le informazioni per comprendere il contributo (che inizierà al Capitolo~\ref{ch:contr} a pagina~\ref{ch:contr}).
	Altra frase di presentazione.

	Nuovo capoverso che parla dello stesso argomento ma è \emph{logicamente} separato
	
	\section{\TeX e \LaTeX}
	In questo capitolo si parla di \TeX e \LaTeX.
	\blindtext
	\subsection{\TeX}
	\Blindtext
	\subsection{\LaTeX}
	\Blindtext
	\subsection*{Altri Materiali}
	\blindtext

	\section{Gli strumenti di Scrittura (Editor)}
	
	\subsection{Distribuzione di \LaTeX}
	\subsubsection{MiK\TeX}
	\subsubsection{\TeX Live}

	\subsection{Editor di Testo più utilizzati}

	%% Troppo dettaglio
	% \subsubsection{Offline}
	% \paragraph{\TeX Works}
	% \paragraph{\TeX Studio}
	% \paragraph{WinEdt}

	% \subsubsection{Online}
	% Qui si parla solo di Overleaf.

	\subsubsection{Windows}
	\paragraph{\TeX Works}
	\paragraph{\TeX Studio}
	\paragraph{WinEdt}
	\subsubsection{Mac OS}
	\paragraph{\TeX Works}
	\paragraph{\TeX Studio}
	\subsubsection{Online (Overleaf)}
	Qui si parla solo di Overleaf



	\section{I Pacchetti più utilizzati}
	\subsection*{hyperref}
	\addcontentsline{toc}{subsection}{hyperref}
	\subsection*{tabularx}
	\addcontentsline{toc}{subsection}{tabularx}
	\subsection*{graphix}
	\addcontentsline{toc}{subsection}{graphix}
	
	\chapter{\label{ch:contr}Contributo}

	Qui di seguito una formula matematica bella da vedere, non come $x^{2} = 4 \rightarrow x = 2$

	\begin{equation}
		\label{eqn:numero1}
		\int_0^1{
			\frac 1 x dx = \left[\log |x|\right]^1_0 = \log 1 - \log 0 = 0 - (- \infty) = +\infty
			}
	\end{equation}
	\begin{equation}
		\label{eqn:numero2}
		\int_2^{+\infty}{
			\frac 1 x dx = \left[\log |x|\right]^{+\infty}_2 = +\infty - \log 2  = +\infty
		}
	\end{equation}
	\begin{equation*}
		x^{x^{x^{x^x}}}
	\end{equation*}

	La prima equazione (Equazione~\ref{eqn:numero1}) 
	mi sta più simpatica della seconda (Equazione~\ref{eqn:numero2})

	\section{Section 1}
	Qui il link alla pagina dell'\href{https://unipr.it}{Università di Parma}

	\subsection{Subsection 1}
	\subsection{Subsection 2}

	\section{Section 2}
	\subsection{Subsection 1}
	\subsubsection{SubSub 1}
	\subsubsection{SubSub 2}
	\subsection{Subsection 2}
	
	\chapter{Metodologia}
	
	\chapter{Conclusioni}

\section{Introduction}

\end{document}