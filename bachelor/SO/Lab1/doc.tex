\documentclass{article}

\author{Leonardo Davoli}
\title{Laboratorio di Sistemi Operativi}

\begin{document}
\maketitle

\section{Esercizi}

\subsection{Esplorare il file system. Qual è il pathname della vostra home directory?}
Per trovare il path della mia home directory devo innanzitutto controllare di trovarmi in essa osservando il prompt della shell.

Una volta che mi trovo in quella cartella uso il comando \texttt{pwd} 
per ottenere stampata la directory corrente su cui stiamo lavorando nella shell.

Il risultato è: \texttt{/home/l.davoli9}

\subsection{Visualizzate i file della home directory ordinati in base alla dimensione.}
Per visualizzare I file nella cartella home in base alla dimensione ho utilizzato il comando \texttt{ls};\\
Le rispettive flag che ho usato sono:
\begin{itemize}
    \item \texttt{-l} --- per mostrare più informazioni sui file tra le quali la dimensione dei file
    \item \texttt{-S} --- per ordinare i file per dimensione dal più grande al più piccolo
    \item \texttt{-r} --- per invertire l'ordine di stampa dei file
\end{itemize}

Il Comando \texttt{ls -Sl} ha restituito il seguente output:
\begin{verbatim}
    -rw-r--r-- 1 UNIPR\l.davoli9 UNIPR\domain^users 603 nov 18 12:17 big_file.txt
    -rw-r--r-- 1 UNIPR\l.davoli9 UNIPR\domain^users 112 nov 18 12:24 medium_file.txt
    -rw-r--r-- 1 UNIPR\l.davoli9 UNIPR\domain^users  14 nov 18 12:16 small_file.txt
\end{verbatim}
mentre il comando \texttt{ls -rSl} ha restituito il seguente output:
\begin{verbatim}
    -rw-r--r-- 1 UNIPR\l.davoli9 UNIPR\domain^users  14 nov 18 12:16 small_file.txt
    -rw-r--r-- 1 UNIPR\l.davoli9 UNIPR\domain^users 112 nov 18  2021 medium_file.txt
    -rw-r--r-- 1 UNIPR\l.davoli9 UNIPR\domain^users 603 nov 18 12:17 big_file.txt
\end{verbatim}

\subsection{Che Differenza c'è tra i comandi \texttt{more}, \texttt{less} seguiti da un nmome di file?}
\texttt{more} e \texttt{less} seguiti dal nome di un file hanno entrambi lo scopo di stampare il contenuto del file a terminale,
simile al comportamento del comando \texttt{cat}, però entrambi troncano il contenuto del file in modo che si possa vedere solo
il contenuto che sta all'interno della finestra del terminale, in entrambi i casi è possibile navigare il file il modo da mostrare
il resto del contenuto a schermo, la differenza tra questi 2 comandi sta nel fatto che il primo permette la navigazione premendo il tasto
\texttt{ENTER} per scorrere verso il basso, mentre il secondo permette la navigazione in entrambi i versi con l'utilizzo delle frecce.

\subsection{Elencare i file contenuti in \texttt{/bin}}
Comando: \texttt{ls -1 /bin > root\_bin\_files.txt}\\
Output: \texttt{root\_bin\_files.txt}\\
In seguito ho spostato il file di output sulla mia macchina locale tramite il comando \texttt{scp} e poi allegato insieme alla relazione.

\subsection{I seguenti comandi che effetto producono? Perché?}
\begin{itemize}
    \item \texttt{cd}: il comando permette di cambiare la directory corrente di lavoro alla propria directory home
    (è un alternativa a \texttt{cd ~})
    \item \texttt{mkdir d1}: crea una directory con nome "d1" all'interno della directory in cui si sta lavorando
    \item \texttt{chmod 444 d1}: Imposta i permessi per tutti gli ruoli: proprietario, gruppo del proprietario e tutti gli altri utenti. 
    Dando solo i permessi di lettura a tutti.\\ 
    Si tratta dei permessi di lettura perché il numero 4 in binario corrisponde a 100 ovvero alla sola attivazione della prima flag
    che corrisponde ai permessi di lettura, essendo il 4 ripetuto 3 volte questa modifica è applicata a tutti e 3 gli ruoli, 
    si può inoltre notare che il permesso di sola lettura per una cartella corrisponde a elencare gli elementi che si trovano dentro quella cartella
    non è perciò possibile ne attraversarla ne modificarla.
    \item \texttt{cd d1} cambia la directory corrente di lavoro alla sottocartella \texttt{d1}, il risultato di questo comando dopo il precedente da
    un errore di permessi negati in quanto il permesso di attraversare e accedere ad una cartella corrisponde alla flag di esecuzione che non è attiva
\end{itemize}

\end{document}
