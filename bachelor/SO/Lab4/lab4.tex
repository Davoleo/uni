\documentclass{article}

\usepackage{listings}
\usepackage{color}
\usepackage{textcomp}
\definecolor{listinggray}{gray}{0.9}
\definecolor{lbcolor}{rgb}{0.9,0.9,0.9}
\lstset{
    backgroundcolor=\color{lbcolor},
    tabsize=4,
    rulecolor=,
    language=make,
    basicstyle=\normalsize\ttfamily,
    upquote=true,
    aboveskip={1.5\baselineskip},
    columns=fixed,
    showstringspaces=false,
    extendedchars=true,
    breaklines=true,
    prebreak = \raisebox{0ex}[0ex][0ex]{\ensuremath{\hookleftarrow}},
    frame=single,
    showtabs=false,
    showspaces=false,
    showstringspaces=false,
    identifierstyle=\ttfamily,
    keywordstyle=\color[rgb]{0,0,1},
    commentstyle=\color[rgb]{0.133,0.545,0.133},
    stringstyle=\color[rgb]{0.627,0.126,0.941},
}

\author{Leonardo Davoli}
\title{Laboratorio di Sistemi Operativi}

\begin{document}
\maketitle

\setcounter{section}{3} % one less
\section{Esercizi}

\subsection{Progettare uno script \texttt{draw\_triangle} che prende in input un parametro intero con valore da 3 a 15 
e disegna sullo standard output un triangolo (utilizzando i caratteri -, / e \textbackslash)}

Script Allegato: \texttt{draw\_triangle.sh}

\subsection{Progettare uno script che prende in input come para\-metro il nome di una directory e conta quanti file hanno dimensione 
$\leq 1KB, \leq 1MB, \leq 1GB, > 1GB$}

Script Allegato: \texttt{file\_sizes.sh}

\subsection{Progettare uno utility-script \texttt{processi} per la gestione user-friendly dei processi in memoria.}

Funzioni dello script:
\begin{enumerate}
    \item Visualizzazione dei processi dell'utente corrente (PID e riga comando)
    \item Eseguire kill su un proprio processo (PID in input)
    \item Eseguire kill -9 su un proprio processo (PID in input)
    \item Mostrare l'elenco degli utenti che hanno almeno un processo attivo nel sistema
\end{enumerate}

Script Allegato: \texttt{processi.sh}

\subsection{Makefile: Introduzione}

Le componenti che creano un \texttt{Makefile} sono:
\begin{itemize}
    \item \textbf{Target}: è l'obbiettivo di una recipe, può essere un file concreto o un concetto più astratto come un cleanup
    \item \textbf{Prerequisite}: sono gli input della ricetta, ovvero i file di cui ha bisogno obbligatoriamente per funzionare (possono anche essere 0)
    \item \textbf{Recipe}: La sequenza di uno o più comandi da eseguire sui prerequisite per ottenere il target
    \item \textbf{Rule}: Spiega quando e come rifare una ricetta su dei requisiti per riottenere un target (ed è il componente base dei Makefile)
\end{itemize}

Un esempio di utilizzo standard di un Makefile può essere per fare il build di un progetto C, attraverso le diverse fasi di compilazione e linking.
Quindi un target ad esempio potrebbero essere i file oggetto mentre i prerequisite potrebbero essere i file sorgente C e gli header:
\begin{lstlisting}
main.o : main.c defs.h
	cc -c main.c
\end{lstlisting}

Esempio di rule di cleanup che rimuove i file oggetto della compilazione precedente
\begin{lstlisting}
clean :
    rm edit main.o kbd.o command.o display.o \
        insert.o search.o files.o utils.o
\end{lstlisting}
\textit{Nota: questa rule ha 0 prerequisite il che significa che non verrà mai eseguita automaticamente da make}

\subsubsection*{Processing dei Makefile}
Un Makefile che viene processato in ordine prende la prima rule (non prefissata da .) 
come 'default goal' questo sarà l'obbiettivo principale del Makefile
\textit{Il 'default goal' si può sovrascrivere cambiando la variabile \texttt{.DEFAULT\_GOAL}}\\

Una volta che \texttt{make} è eseguito nella directory del progetto ed processa la prima regola.
Se la prima regola necessita di file (come prerequisiti) 
per i quali sono presenti altre regole esegue anche quest'ultime.\\
Se una regola non ha una dipendenza sul default goal essa non è eseguita automaticamente da make.

Make riesegue la regola relativa al default goal ogni volta che ad esempio il file eseguibile 
non esiste oppure è più vecchio rispetto ad uno dei file oggetto compilati 
o dei file sorgente modificati; in questi casi make riesegue le recipe necessarie per il global goal 
e infine riesegue la task principale.

\end{document}