\documentclass{article}

\begin{document}
\setcounter{section}{2} % one less
\section{Laboratorio - Esercizi}

\subsection{Costruire una manipolazione dell'output di \texttt{ps} che ordini i processi per PID}
Comando: \texttt{ps | tr -s ' ' | egrep \^{}\symbol{92} [0\-9]+ | sort -t' ' -k1 -r}
\begin{itemize}
    \item \texttt{ps} ottiene la lista di processi dell'utente corrente
    \item \texttt{tr -s ' '} rimuove gli spazi in più compressandoli in uno unico
    \item \texttt{egrep \^{}\symbol{92}\space[0-9]+} filtra solo le linee che iniziano con uno spazio e il PID (rimuove l'header)
    \item \texttt{sort -t' ' -k1} ordina i processi per il primo campo con delimitatore '\space' 
    (ho aggiunto -r in modo che siano ordinati in senso decrescente visto che il comando ps di default ordina i processi per PID crescente)
\end{itemize}

\subsection{Fare alcuni esperimenti per scoprire qual è l'effetto del comando 
\texttt{tr str1 str2} se le 2 stringhe hanno lunghezze diverse.}
Nel caso in cui la prima stringa sia più piccola della seconda i caratteri extra nella seconda non verranno 
mai utilizzati da tr durante la sostituzione; Mentre nel caso la seconda stringa sia più piccola della prima i caratteri extra della prima 
stringa vengono sostituiti tutti con l'ultimo carattere della prima stringa.

\subsection{Scrivere un comando per sostituire modificare l'output di \texttt{ls -l} 
in modo che al posto degli spazi sia mostrato un carattere \texttt{<Tab>} 
(non devono comparire più \texttt{<Tab>} consecutivi)}
Comando: \texttt{ls -l | tr -s ' ' '\symbol{92}t' > out}\\
Il comando permette comprime gli spazi di divisione tra gli elementi dell'output ad 1 spazio e li sostituisce con \textbackslash t 
la sequenza di escape che produce il carattere \texttt{<Tab>}, in questo modo tutti i campi dell'output sono divisi da 1 e un solo tab.
\textit{Nota: Ho ridirezionato l'output ad un file esterno perché il terminale mi mostrava gli spazi nonostante avessi applicato il comando tr}

\subsection{Scrivere una pipeline che permetta di scoprire quante linee ripetute ci sono in un file.}
Comando: 


\end{document}