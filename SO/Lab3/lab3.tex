\documentclass{article}

\usepackage{listings}
\usepackage{color}
\usepackage{textcomp}
\definecolor{listinggray}{gray}{0.9}
\definecolor{lbcolor}{rgb}{0.9,0.9,0.9}
\lstset{
	backgroundcolor=\color{lbcolor},
	tabsize=4,
	rulecolor=,
	language=awk,
        basicstyle=\normalsize\ttfamily,
        upquote=true,
        aboveskip={1.5\baselineskip},
        columns=fixed,
        showstringspaces=false,
        extendedchars=true,
        breaklines=true,
        prebreak = \raisebox{0ex}[0ex][0ex]{\ensuremath{\hookleftarrow}},
        frame=single,
        showtabs=false,
        showspaces=false,
        showstringspaces=false,
        identifierstyle=\ttfamily,
        keywordstyle=\color[rgb]{0,0,1},
        commentstyle=\color[rgb]{0.133,0.545,0.133},
        stringstyle=\color[rgb]{0.627,0.126,0.941},
}

\author{Leonardo Davoli}
\title{Laboratorio di Sistemi Operativi}

\begin{document}
\maketitle

\setcounter{section}{2} % one less
\section{Esercizi}

\subsection{Costruire una manipolazione dell'output di \texttt{ps} che ordini i processi per PID}
Comando: \texttt{ps | tr -s ' ' | egrep \^{}\symbol{92} [0-9]+ | sort -t' ' -k1 -r}
\begin{itemize}
    \item \texttt{ps} ottiene la lista di processi dell'utente corrente
    \item \texttt{tr -s ' '} rimuove gli spazi in più compressandoli in uno unico
    \item \texttt{egrep \^{}\symbol{92}\space[0-9]+} filtra solo le linee che iniziano con uno spazio e il PID (rimuove l'header)
    \item \texttt{sort -t' ' -k1} ordina i processi per il primo campo con delimitatore '\space' 
    (ho aggiunto -r in modo che siano ordinati in senso decrescente visto che il comando ps di default ordina i processi per PID crescente)
\end{itemize}

\subsection{Fare alcuni esperimenti per scoprire qual è l'effetto del comando 
\texttt{tr str1 str2} se le 2 stringhe hanno lunghezze diverse.}
Nel caso in cui la prima stringa sia più piccola della seconda i caratteri extra nella seconda non verranno 
mai utilizzati da tr durante la sostituzione; Mentre nel caso la seconda stringa sia più piccola della prima i caratteri extra della prima 
stringa vengono sostituiti tutti con l'ultimo carattere della prima stringa.

\subsection{Scrivere un comando per sostituire modificare l'output di \texttt{ls -l} 
in modo che al posto degli spazi sia mostrato un carattere \texttt{<Tab>} 
(non devono comparire più \texttt{<Tab>} consecutivi)}
Comando: \texttt{ls -l | tr -s ' ' '\symbol{92}t' > out}\\
Il comando permette comprime gli spazi di divisione tra gli elementi dell'output ad 1 spazio e li sostituisce con \textbackslash t 
la sequenza di escape che produce il carattere \texttt{<Tab>}, in questo modo tutti i campi dell'output sono divisi da 1 e un solo tab.
\textit{Nota: Ho ridirezionato l'output ad un file esterno perché il terminale mi mostrava gli spazi nonostante avessi applicato il comando tr}

\subsection{Scrivere una pipeline che permetta di scoprire quante linee ripetute ci sono in un file.}
Comando: \texttt{cat file.txt | sort -t\symbol{92}n | uniq -d | wc -l}
\begin{itemize}
    \item \texttt{text -t\symbol{92}n} permette di riordinare le righe in ordine alfabetico per permettere a uniq di lavorare correttamente
    \item \texttt{uniq -d} restituisce tutti i duplicati nell'input presi una volta ciascuno
    \item \texttt{wc -l} conta tutte le righe dell'input (in questo caso solo quelle duplicate)
\end{itemize}

\subsection{Visualizzare su standard output, senza ripetizioni, lo user ID di tutti gli utenti che hanno almeno un processo attivo nel sistema}
Comando: \texttt{ps -el | tr -s ' ' | cut -d' ' -f3 | egrep [0-9]+ | sort -n -t\symbol{92}n | uniq}\\
Output (sulla VM di didattica):
\begin{verbatim}
    0
    33
    100
    101
    102
    103
    62583
    605561111
    1579333809
    1579341764
    1579346453
    1579348286
    1579353610
    1579356982
    1579357026
    1579360847
    1579361305
\end{verbatim}

\subsection{Scrivere un comando awk per fare una statistica di un file in put per contare quante righe contengono 0, 1, 2 e 3 parole.}
Comando: \texttt{awk -f count.awk file.txt}\\\\

Contenuto di count.awk:
\begin{lstlisting}[numbers=left]
    BEGIN {
        print "scanning lines: "
    }

    {
        printf "%d...", NR;
        if (NF == 0)
            zeroFieldLines++;
        if (NF == 1)
            oneFieldLines++;
        if (NF == 2)
            twoFieldLines++;
        if (NF == 3)
            threeFieldLines++;
    }

    END {
        print "\n\n"
        printf "0 words lines: %d\n1 word lines: %d\n2 words lines: %d\n3 words lines: %d\n", zeroFieldLines, oneFieldLines, twoFieldLines, threeFieldLines;
    }
\end{lstlisting}

\end{document}