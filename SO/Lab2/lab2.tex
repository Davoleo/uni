\documentclass{article}

\author{Leonardo Davoli}
\title{Laboratorio di Sistemi Operativi}

\begin{document}
\maketitle

\setcounter{section}{1} % one less
\section{Esercizi}

\subsection{Scrivere un comando (con pipe) per:}
\begin{itemize}
    \item Contare quanti studenti hanno un account sulla macchina (home directory):
    \begin{itemize}
        \item Comando: \texttt{cd /home ; ls -1 | wc -l}
        \item Output: \texttt{917}
    \end{itemize}
    \item Contare quanti file sono presenti nella home:
    \begin{itemize}
        \item Comando: \texttt{cd ; ls -l | grep -c 1\symbol{92} UNIPR}
        \item Output: \texttt{5}
    \end{itemize}
    \item lista dei file in \texttt{/dev} che sono composti da due caratteri alfanumerici seguiti da una cifra 0-9:
    \begin{itemize}
        \item Comando: \texttt{cd /dev ; ls -1 | grep ??[0-9]}
        \item Output: \texttt{grep: sg0: Permesso negato}
    \end{itemize}
\end{itemize}

\subsection{Qual è la differenza tra i seguenti comandi?}
\begin{itemize}
    \item \texttt{ls > ls}: Ridirezione l'output di \texttt{ls} nella cartella corrente in un file di nome "ls"
    \item \texttt{ps | ls}: Esegue il comando \texttt{ps} passando l'output al comando \texttt{ls}
    \item \texttt{ls | ps}: Esegue il comando \texttt{ls} passando l'output al comando \texttt{ps}
\end{itemize}

\subsection{Quale effetto producono i seguenti comandi? }
\begin{itemize}
    \item \texttt{cat file | sort | uniq}, dove "file" è il nome di un file: Stampa il contenuto del file ordinandolo secondo l'ordine naturale
    e rimuovendo le linee duplicate
    \item \texttt{who | wc -l}: Conta tutti gli utenti connessi alla macchina e stampa il risultato a schermo.
    \item \texttt{ps -e | wc -l}: Conta tutti i processi attivi in quel momento.
\end{itemize}

\subsection{Definire il comando \texttt{ll} in modo che venga chiamato \texttt{ls -l}}
Comando:
\begin{verbatim}
    alias ll='ls -l'
\end{verbatim}

\subsection{Quale stringa devo impostare per avere un prompt che indichi il path completo?}
Devo impostare la stringa: \texttt{"\symbol{92}w>"} nella variabile \texttt{PS1}

\subsection{Stampare a video l'ultimo comando lanciato}
Comando: \texttt{history | tail -2 | head -1 | cut -d\symbol{92}\space -f5-}\\
Il comando cut come delimitatore prende '\textbackslash\space' che sarebbe uno spazio protetto da escape character,
in seguito prendo dal quinto field fino alla fine tramite la flag \texttt{-f} con valore \texttt{5-}, 
si tratta del quinto campo perché l'indice del comando nella history è circondato da 2 spazi sia prima che dopo.

\subsection{Scrivere un comando che fornisca il numero dei comandi distinti contenuti nella history list.}
Comando: \texttt{history | cut -d\symbol{92}\space -f5- | sort | uniq | wc -l}\\
Sfrutto il comando precedente che rimuoveva l'indice di comando della history tramite \texttt{cut}, rimuovendo i comandi \texttt{head e tail}
utilizzati per restringere la stampa all'ultimo comando inserito, aggiungendo \texttt{sort e uniq} per riordinare e rimuovere linee duplicate e 
poi contandole con \texttt{wc -l}

\subsection{Come si lancia un eseguibile solo se la compilazione ha avuto successo?}
Comando: \texttt{gcc test.c -o test\_compiled \&\& ./test\_compiled}\\
\textit{Nota: Il comando utilizza GCC come compilatore d'esempio e un semplice file c con dummy code}

\subsection{Elencare tutti i vostri file che contengono una 'a'.}
Comando: \texttt{ls *a*}\\
Il metacarattere '*' è una wildcard per 0 o più caratteri.

\subsection{Qual è l'effetto dei seguenti comandi?}
\begin{itemize}
    \item \texttt{ls -R || (echo file non accessibili > tmp)}:\\ elenca gli elementi della cartella corrente
    e di tutte le sottocartelle ricorsivamente, nel caso il comando fallisca 
    (probabilmente per una mancanza di permessi di lettura/esecuzione sulle cartelle), scrive dentro a al file tmp "file non accessibili"
    \item \texttt{cat /etc/issue > temp.txt; cat /etc/issue >> temp.txt;\\ cat temp.txt | uniq}:
    Sequenza di 3 comandi:
    \begin{itemize}
        \item Il primo estrae il contenuto del file \texttt{/etc/issue} e lo scrive dentro il file \texttt{temp.txt} nella cartella stessa
        \item Il secondo estrae il contenuto dallo stesso file ma invece di\\ sovrascriverlo al precedente dentro \texttt{temp.txt} lo appende 
        alla fine di esso (\texttt{>>})
        \item Il terzo comando estrae il contenuto da \texttt{temp.txt} ridirezionandolo in input a uniq che però non cancella 
        le righe duplicate in quanto le righe identiche non sono adiacenti nel file.
    \end{itemize}
    \item \texttt{(cd /bin ; pwd ; ls | wc -l)}:
    Sequenza di 3 comandi:
    \begin{itemize}
        \item Il primo cambia la directory di lavoro corrente a \texttt{/bin}
        \item Il secondo stampa la directory di lavoro corrente su terminale (sarà \texttt{/bin})
        \item Il terzo elenca gli elementi di quella directory ridirezionando l'output a \texttt{wc -l} 
        che, a sua volta, ritornerà il numero di linee quindi il numero di file in /bin
    \end{itemize}
    \textit{Tutti i 3 comandi sono raggruppati quindi il comando \texttt{cd} 
    non avrà side-effect sul terminale dopo che l'esecuzione dei 3 comandi è finita}
\end{itemize}

\end{document}