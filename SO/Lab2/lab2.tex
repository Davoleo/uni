\documentclass{article}

\begin{document}
\setcounter{section}{1} % one less
\section{Laboratorio - Esercizi}

\subsection{Scrivere un comando (con pipe) per:}
\begin{itemize}
    \item Contare quanti studenti hanno un account sulla macchina (home directory):
    \begin{itemize}
        \item Comando: \texttt{cd /home ; ls -1 | wc -l}
        \item Output: \texttt{917}
    \end{itemize}
    \item Contare quanti file sono presenti nella home:
    \begin{itemize}
        \item Comando: \texttt{cd ; ls -l | grep -c 1\symbol{92} UNIPR}
        \item Output: \texttt{5}
    \end{itemize}
    \item lista dei file in \texttt{/dev} che sono composti da due caratteri alfanumerici seguiti da una cifra 0-9:
    \begin{itemize}
        \item Comando: \texttt{cd /dev ; ls -1 | grep ??[0-9]}
        \item Output: \texttt{grep: sg0: Permesso negato}
    \end{itemize}
\end{itemize}

\subsection{Qual è la differenza tra i seguenti comandi?}
\begin{itemize}
    \item \texttt{ls > ls}: Ridirezione l'output di \texttt{ls} nella cartella corrente in un file di nome "ls"
    \item \texttt{ps | ls}: Esegue il comando \texttt{ps} passando l'output al comando \texttt{ls}
    \item \texttt{ls | ps}: Esegue il comando \texttt{ls} passando l'output al comando \texttt{ps}
\end{itemize}

\subsection{Quale effetto producono i seguenti comandi? }
\begin{itemize}
    \item \texttt{cat file | sort | uniq}, dove "file" è il nome di un file: Stampa il contenuto del file ordinandolo secondo l'ordine naturale
    e rimuovendo le linee duplicate
    \item \texttt{who | wc -l}: Conta tutti gli utenti connessi alla macchina e stampa il risultato a schermo.
    \item \texttt{ps -e | wc -l}: Conta tutti i processi attivi in quel momento.
\end{itemize}

\subsection{Definire il comando \texttt{ll} in modo che venga chiamato \texttt{ls -l}}
Comando:
\begin{verbatim}
    alias ll='ls -l'
\end{verbatim}

\subsection{Quale stringa devo impostare per avere un prompt che indichi il path completo?}
Devo impostare la stringa: \texttt{"\symbol{92}w>"} nella variabile \texttt{PS1}

\subsection{Stampare a video l'ultimo comando lanciato}
Comando: \texttt{history | tail -2 | head -1}\\
\textit{Nota: Mantiene il numero dietro al nome del comando di quando si chiama il comando history.}

\subsection{Scrivere un comando che fornisca il numero dei comandi distinti contenuti nella history list.}

\end{document}